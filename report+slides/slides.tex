\documentclass{beamer}
\usetheme{Madrid}
\usecolortheme{default}
\usepackage{graphicx}

\title{Florida Flood Gauge Monitor}
\subtitle{Real-Time Interactive Web Application with AI Analysis}
\author{Paul Fishwick and Claude Code}
\date{\today}

\begin{document}

\frame{\titlepage}

\begin{frame}{Project Overview}
\begin{itemize}
\item Web-based application for real-time flood gauge monitoring across Florida
\item Integrates live data from USGS Water Services API
\item Interactive map displaying active monitoring stations
\item AI-powered explanations of water level fluctuations
\item Built with modern web technologies (React, TypeScript, Vite)
\end{itemize}

\vspace{0.3cm}
\textbf{Purpose:} Transform raw sensor data into actionable insights accessible to both technical and non-technical users
\end{frame}

\begin{frame}{Interface Demonstration}
\begin{figure}
\centering
\includegraphics[width=0.95\textwidth]{interface.png}
\caption{Interactive map with gauge locations, historical data chart, and AI-generated analysis panel}
\end{figure}
\end{frame}

\begin{frame}{Key Features}
\begin{columns}
\begin{column}{0.5\textwidth}
\textbf{Data Integration}
\begin{itemize}
\item Live USGS API data
\item 7-day historical trends
\item Active gauges statewide
\item Real-time updates
\end{itemize}

\vspace{0.3cm}
\textbf{Visualization}
\begin{itemize}
\item Interactive OpenStreetMap
\item Time-series charts
\item Responsive design
\item Intuitive navigation
\end{itemize}
\end{column}

\begin{column}{0.5\textwidth}
\textbf{AI-Powered Analysis}
\begin{itemize}
\item Google Gemini integration
\item Three detail levels
\item Contextual explanations
\item Cached for performance
\end{itemize}

\vspace{0.3cm}
\textbf{Factors Analyzed}
\begin{itemize}
\item Water body type
\item Tidal influences
\item Precipitation patterns
\item Regional topography
\end{itemize}
\end{column}
\end{columns}
\end{frame}

\begin{frame}{System Architecture}
\textbf{Component-Based Design:}
\begin{itemize}
\item \textbf{App Component:} State management, data orchestration, error handling
\item \textbf{Map Component:} Leaflet-based interactive mapping with gauge markers
\item \textbf{GaugeChart Component:} Recharts visualization of historical water levels
\item \textbf{GaugeStory Component:} AI-generated explanations with detail toggles
\end{itemize}

\vspace{0.3cm}
\textbf{Unidirectional Data Flow:}
\begin{enumerate}
\item Fetch USGS data on initialization
\item Store in component state
\item Pass to child components via props
\item User interactions update parent state
\item Components re-render with new data
\end{enumerate}
\end{frame}

\begin{frame}{Technical Stack}
\begin{table}
\begin{tabular}{ll}
\textbf{Category} & \textbf{Technology} \\
\hline
Framework & React 19 with TypeScript 5 \\
Build Tool & Vite 5 (fast development server) \\
Mapping & Leaflet 1.9 + React-Leaflet 5 \\
Charts & Recharts 3 \\
AI Integration & Google Gemini API (gemini-2.5-flash) \\
Data Source & USGS Water Services API \\
Deployment & Single-page application (SPA) \\
\end{tabular}
\end{table}

\vspace{0.3cm}
\textbf{Key Implementation Details:}
\begin{itemize}
\item Environment variables via Vite (\texttt{import.meta.env})
\item Story caching for instant detail level switching
\item Comprehensive error handling and loading states
\end{itemize}
\end{frame}

\begin{frame}{Conclusion}
\textbf{Achievements:}
\begin{itemize}
\item Successfully integrated real-time government data with modern web interface
\item Enhanced raw sensor data with AI-generated contextual explanations
\item Created accessible tool for both technical and non-technical users
\item Demonstrated effective use of React, TypeScript, and AI technologies
\end{itemize}

\vspace{0.3cm}
\textbf{Future Enhancements:}
\begin{itemize}
\item Additional data sources (weather, precipitation forecasts)
\item Predictive flood modeling capabilities
\item Extended historical trend analysis
\item Mobile-optimized responsive interface
\item User alerts for critical water level changes
\end{itemize}
\end{frame}

\end{document}
