\documentclass[11pt]{article}
\usepackage[margin=1in]{geometry}
\usepackage{graphicx}
\usepackage{hyperref}
\usepackage{xcolor}
\usepackage{enumitem}

\hypersetup{
    colorlinks=true,
    linkcolor=blue,
    urlcolor=blue
}

\title{Florida Flood Gauge Monitor\\[0.5em]\large Documentation}
\author{Paul Fishwick and Claude Code}
\date{February 2026}

\begin{document}

\maketitle

\section{Overview}

The Florida Flood Gauge Monitor is a web application that displays real-time water level data from over 500 USGS monitoring stations across Florida. Users can explore an interactive map, view 7-day historical charts for individual gauges, and generate AI-powered narratives explaining water level behavior.

\section{Architecture}

The application is built with the following technologies:

\begin{itemize}[nosep]
    \item \textbf{React + TypeScript} -- UI framework
    \item \textbf{Vite} -- Build tool and dev server
    \item \textbf{Leaflet} -- Interactive map rendering via OpenStreetMap
    \item \textbf{Recharts} -- Time series chart visualization
    \item \textbf{Google Gemini API} -- AI-generated gauge stories
    \item \textbf{USGS Water Services API} -- Live gauge data source
\end{itemize}

\section{Features}

\subsection{Interactive Map}

The main view displays all active USGS flood gauges in Florida as blue markers on an OpenStreetMap base layer, as shown in Figure~\ref{fig:map}. Users can pan and zoom to explore gauge locations across the state.

\begin{figure}[ht]
    \centering
    \includegraphics[width=\textwidth]{map-overview.png}
    \caption{Main map view showing 557 active USGS flood gauges across Florida.}
    \label{fig:map}
\end{figure}

\subsection{Gauge Chart}

Clicking on any gauge marker opens a bottom panel displaying a 7-day historical chart of water level changes, as shown in Figure~\ref{fig:chart}. The y-axis shows water level change in feet, and the x-axis shows the date range. A ``Story'' button is available to generate an AI explanation.

\begin{figure}[ht]
    \centering
    \includegraphics[width=\textwidth]{gauge-chart.png}
    \caption{Historical water level chart for a selected gauge, with the Story button visible.}
    \label{fig:chart}
\end{figure}

\subsection{AI-Generated Story}

The Story feature uses Google Gemini to generate a narrative explanation of the water level data, as shown in Figure~\ref{fig:story}. Three detail levels are available:

\begin{itemize}[nosep]
    \item \textbf{Summary} -- One to two sentences
    \item \textbf{Standard} -- A single paragraph
    \item \textbf{Detailed} -- Two to three paragraphs
\end{itemize}

All three levels are fetched concurrently and cached, so switching between them is instant.

\begin{figure}[ht]
    \centering
    \includegraphics[width=\textwidth]{gauge-story.png}
    \caption{AI-generated story modal explaining water level behavior for a selected gauge.}
    \label{fig:story}
\end{figure}

\section{Setup}

\subsection{Prerequisites}

\begin{itemize}[nosep]
    \item Node.js (v18+)
    \item npm
\end{itemize}

\subsection{Installation}

\begin{verbatim}
npm install
\end{verbatim}

\subsection{API Key Configuration}

The Story feature requires a free Google Gemini API key. Obtain one at \url{https://aistudio.google.com}, then create a \texttt{.env.local} file in the project root:

\begin{verbatim}
VITE_GEMINI_API_KEY=your_key_here
\end{verbatim}

The application functions fully without an API key; only the Story feature is unavailable.

\subsection{Running}

\begin{verbatim}
./start
\end{verbatim}

This script kills any existing process on port 5173, starts the Vite dev server, and opens the browser.

\section{License}

This project is released under the MIT License.

\end{document}
